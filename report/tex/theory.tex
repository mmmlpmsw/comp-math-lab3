\newpage

\title{Описание использованного метода}{\begin{center}
    Описание метода
\end{center}} \\
Некоторая функция $f(x)$ задана на отрезке $[a, b]$, разбитом на части $[x_{i − 1}, x_i]$, таком, что
$a = x_0 < x_1 < . . . < x_n = b$. Кубическим сплайном называется функция $S(x)$, которая
\newline
– на каждом отрезке $[ x_{i − 1}, x_i]$ является многочленом степени не выше третьей;
\newline
– имеет непрерывные первую и вторую производные на всём отрезке $[a,b]$;
\newline
– в точках $x_i$ выполняется равенство $S(x_i)=f(x_i)$, т. е. сплайн $S(x)$ интерполирует функцию f в
точках $x_i$. \\
Интерполяция кубическими сплайнами - частный случай кусочно-полиноминальной интерполяции. В данном случае между любыми двумя соседними узлами функция интерполируется кубическим полиномом. Его коэффициенты на каждом интервале определяются из условий сопряжения в узлах: \\
$$f_{i}=y_{i}, f^{'}(x_{i}-0)=f^{'}(x_{i}+0), f^{''}(x_{i}-0)=f^{''}(x_{i}+0), i=1,2,...,n-1. $$ \\
Кроме того, на границе при $x=x_{0}$ и $x=x_{n}$ ставятся условия:
$$f''(x_0)=0, f''(x_n)=0. \eqno(2)$$ 
Будем искать кубический полином в виде
$$f(x)=a_i+b_i(x-x_{i-1})+c_i(x-x_{i-1})^2+d_i(x-x_{i-1})^3, x_{i-1}\le \xi\le \xi_i.\eqno(3)$$
Из условия $f_i=y_i$ имеем
$$f(x_{i-1})=a_i=y_{i-1},
f(x_i)=a_i+b_ih_i+c_ih_i^2+d_ih_i^3=y_i,
h_i=x_i-x_{i-1}, i=1, 2, \cdots, n-1. \eqno(4)$$
Вычислим производные:
$$f'(x)=b_i+2c_i(x-x_{i-1})+3d_i(x-x_{i-1})^2,
f''(x)=2c_i+6d_i(x-x_{i-1}), x_{i-1}\le \xi\le \xi_i, $$
и потребуем их непрерывности при $x=x_i$:
$$b_{i+1}=b_i+2c_ih_i + 3d_ih_i^2,
c_{i+1}=c_i+3d_ih_i, i=1, 2, \cdots, n-1. \eqno(5)$$
Общее число неизвестных коэффициентов, очевидно, равно $4n$, число уравнений (4) и (5) равно $4n-2$. Недостающие два уравнения получаем из условия (2) при $x=x_0$ и $x=x_n$:
$$c_1=0, c_n+3d_nh_n=0.$$
Выражение из (5) $d_i=\frac{c_{i+1}-c_i}{3h_i}$, подставляя это выражение в (4) и исключая $a_i=y_{i-1}$, получим
$$b_i=\frac{y_i-y_{i-1}}{h_{i}}\--\frac{1}{3}h_i(c_{i+1}+2c_i),  i=1, 2, \cdots, n-1,
b_n=\frac{y_n-y_{n-1}}{h_n}\--\frac{2}{3}h_nc_n.$$
Подставив теперь выражения для $b_i, b_{i+1}$ и $d_i$ в первую формулу (5), после несложных преобразований получаем для определения $c_i$ разностное уравнение второго порядка

$$h_ic_i+2(h_i+h_{i+1})c_{i+1}+h_{i+1}c_{i+2}=3\left(\frac{y_{i+1}-y_i}{h_{i+1}} - \frac{y_i-y_{i-1}}{h_i}\right), i=1, 2, \cdots, n-1.\eqno(6)$$

С краевыми условиями
$$c_1=0, c_{n+1}=0 \eqno(7) $$

Условие $c_{n+1}=0$ эквивалентно условию $c_n+3d_nh_n=0$ и уравнению $c_{i+1} = c_i+d_ih_i$. Разностное уравнение (6) с условиями (7) можно решить методом прогонки, представив в виде системы линейных алгебраических уравнений вида $~A*x=F$, где вектор $x$ соответствует вектору $\{c_i\}$, вектор $F$ поэлементно равен правой части уравнения (6), а матрица $~A$ имеет следующий вид:

$$A = \begin{pmatrix} C_1 & B_1 & 0   & 0   & \cdots & 0 & 0
                         \\ A_2 & C_2 & B_2 & 0   & \cdots & 0 & 0
                         \\ 0   & A_3 & C_3 & B_3 & \cdots & 0 & 0 
                         \\ \cdots & \cdots & \cdots & \cdots & \cdots & \cdots & \cdots 
                         \\ \cdots & \cdots & \cdots & \cdots & \cdots & \cdots & \cdots 
                         \\ \cdots & \cdots & \cdots & \cdots & \cdots & \cdots & B_{n-1}
                         \\ 0 & 0 & 0 & 0 & \cdots & A_{n} & C_{n}
            \end{pmatrix}, $$
где $A_i=h_i,  i=2, \cdots, n,  B_i = h_{i+1},  i=1, \cdots, n-1$\\ и $C_i=2(h_i+h_{i+1}), i =1, \cdots, n. $ \\

\begin{center}
    Метод прогонки
\end{center}
Метод прогонки, основан на предположении, что искомые неизвестные связаны рекуррентным соотношением:
$$x_i = \alpha_{i+1}x_{i+1} + \beta_{i+1}\, i=1,\cdots,n-1  \eqno(8) $$
Используя это соотношение, выразим $x_{i-1}$ и $x_i$ через $x_{i+1}$ и подставим в i-e уравнение:
$$\left(A_i\alpha_i\alpha_{i+1} + C_i\alpha_{i+1} + B_i\right)x_{i+1} + A_i\alpha_i\beta_{i+1} + A_i\beta_i + C_i\beta_{i+1} - F_i = 0$$,
где $F_i$ - правая часть i-го уравнения. Это соотношение будет выполняться независимо от решения, если потребовать
$$A_i\alpha_i\alpha_{i+1} + C_i\alpha_{i+1} + B_i = 0$$
$$A_i\alpha_i\beta_{i+1} + A_i\beta_i + C_i\beta_{i+1} - F_i = 0$$
Отсюда следует:
$$\alpha_{i+1} = {-B_i \over A_i\alpha_i + C_i}\
\beta_{i+1} = {F_i - A_i\beta_i \over A_i\alpha_i + C_i} $$
Из первого уравнения получим:
$$\alpha_2 = {-B_1 \over C_1}
\beta_2 = {F_1 \over C_1}$$
После нахождения прогоночных коэффициентов $\alpha$ и $\beta$, используя уравнение (1), получим решение системы. При этом,
$$x_n = {F_n-A_n\beta_n \over C_n+A_n\alpha_n} $$.